\documentclass[10pt,usenames,dvipsnames,professionalfonts]{beamer}

\usepackage{beamerthemepecostalk}

\usefonttheme[onlymath]{serif}
%\usefonttheme[onlylarge]{structurebold}
%\usefonttheme{default}

\beamertemplatenavigationsymbolsempty

%\setbeamerfont{itemize/enumerate subbody}{size=\normalsize} %to set the body size
%\setbeamertemplate{itemize subitem}{\normalsize\raise1.25pt\hbox{\donotcoloroutermaths$\blacktriangleright$}}  %to set the symbol size



\usepackage{etex}
\usepackage{tikz}
\usepackage{pgfplots}

\usepackage{tcolorbox}
\usepackage{multirow}


%\newtcolorbox{mybox}[1][Theorem:]{
\newtcolorbox{mybox}{
colback=white,
colbacktitle=white,
coltitle=blue!70!black,
%colframe=\color{pecos1},%red!70!black,
colframe=pecos1,
boxrule=1.5pt,
titlerule=0pt,
%arc=15pt,
%title={\strut#1}
}

\newtcolorbox{myboxth}[1][Theorem:]{
colback=white,
colbacktitle=white,
coltitle=pecos1,
colframe=pecos1,
boxrule=1.5pt,
titlerule=0.5pt,
%arc=15pt,
title={\strut#1}
}


%\usepackage{pxfonts}
%\usepackage{eulervm}



\usepackage{mathpazo}


%\usepackage[T1]{fontenc}
%\usepackage{fourier}
%\usepackage[charter]{mathdesign}

%\usepackage{epsfig}
\usepackage{epstopdf}
\usepackage{bbm}
\usepackage{longtable}
\usepackage{ifthen}
\usepackage{listings}
\usepackage{boxedminipage}
\usepackage{colortbl}
\usepackage{array}
\usepackage{alltt}
\usepackage{esint}

%\usepackage{extsize}
\usepackage{synttree}
\usepackage{booktabs}
\usepackage{tabularx}
\usepackage{colortbl}
\usepackage{xcolor}
\usepackage[slide]{algorithm2e}
%\usepackage[plain]{algorithm}
%\usepackage{algorithmic}
\usepackage[english]{babel}
%\usepackage{makeidx}
\usepackage{amsthm}
\usepackage{mathtools}
\usepackage{amsmath,amssymb}
\usepackage{mathrsfs}
\usepackage{tikz}
\usepackage{tikz-cd}
\usepackage[makeroom]{cancel}
\usepackage{multimedia}
\usepackage{amscd}
\usepackage{latexsym}
\usepackage{fancyvrb}
\usepackage{graphics}
\usepackage{relsize}
\usepackage{subfigure}
\usepackage{soul}
\usepackage{pgf}
%\usepackage{bm}
%\usepackage{caption}
\usepackage[labelformat=empty]{caption}
\setbeamertemplate{caption}{\raggedright\insertcaption\par}
%\captionsetup{labelformat=empty,labelsep=none}
\renewcommand*{\thesubfigure}{}

\SetKwComment{Comment}{${\color{red}\triangleright}\:$}{}
%\SetKwComment{Comment}{}{}

\swapnumbers

\setbeamercovered{transparent=30}

\branchheight{.4in}



\newtheorem{teorema}{Theorem}
\newtheorem{proposition}[theorem]{Proposition}
\newtheorem{remark}[theorem]{Remark}
%\newtheorem{lemma}[theorem]{Lemma}
\newtheorem{definizione}[theorem]{Def\mbox{}inition}
%\newtheorem{problem}[theorem]{Problem}

%justified text
%\renewcommand{\raggedright}{\leftskip=0pt \rightskip=0pt plus 0cm} 

%\usepackage{ragged2e}
%\justifying


\newcommand{\jump}[1]{\ensuremath{[\![#1]\!]} }
\newcommand{\dive}{\operatorname{div}}
\newcommand{\grad}{\operatorname{grad}}
\newcommand{\Cof}{\operatorname{Cof}}
\newcommand{\id}{\operatorname{id}}
\newcommand{\sign}{\operatorname{sign}}
\newcommand{\vol}{\operatorname{vol}}
\newcommand{\area}{\operatorname{area}}
\newcommand{\spa}{\operatorname{span}}
\newcommand{\rank}{\operatorname{rank}}
\newcommand{\curl}{\operatorname{curl}}
\newcommand{\inte}{\operatorname{int}}
\newcommand{\diam}{\operatorname{diam}}
\newcommand{\diag}{\operatorname{diag}}
\newcommand{\meas}{\operatorname{meas}}
\newcommand{\supp}{\operatorname{supp}}
\newcommand{\bcurl}{\operatorname{\mathbf{curl}}}
\renewcommand{\deg}{\operatorname{\mathit{deg}}}
\newcommand{\vv}{\boldsymbol{v}}
\newcommand{\f}{\boldsymbol{{f}}}
\newcommand{\n}{\boldsymbol{{n}}}
\renewcommand{\u}{\boldsymbol{{u}}}
\newcommand{\e}{\boldsymbol{{e}}}
\newcommand{\bnu}{\boldsymbol{{\nu}}}
\newcommand{\x}{\boldsymbol{\mathrm{x}}}
\providecommand{\E}[1]{\ensuremath{\,\text{E}{#1}}}
%\renewcommand{\CancelColor}{\blue}

%\renewcommand\bcancel[2][black]{\renewcommand\CancelColor{\color{#1}}\bcancel{#2}}


%\algsetup{indent=2em}
%\renewcommand{\qed}{\hf\mbox{}ill \mbox{\raggedright \rule{.07in}{.1in}}}


\newcommand{\blob}{
 \rule[.0ex]{0.8ex}{0.8ex}
}
\renewcommand{\qedsymbol}{\blob}

\newcommand\textbox[1]{%
  \parbox{.5\textwidth}{#1}%
}

%\newif\ifpdf
%\ifx\pdfoutput\undefined
%\pdffalse % we are not running PDFLaTeX
%\else
%\pdfoutput=1 % we are running PDFLaTeX
%\pdftrue
%\fi
%\ifpdf
%\usepackage[pdftex]{graphicx}
%\else
%\usepackage{graphicx}
%\fi


\SetKwBlock{Do}{do}{enddo}
\SetKwInput{KwInOut}{Input/Output}



\renewcommand{\arraystretch}{1.2}

\makeatletter
\newcommand{\thickhline}{%
    \noalign {\ifnum 0=`}\fi \hrule height 1.5pt
    \futurelet \reserved@a \@xhline
}
\newcolumntype{"}{@{\hskip\tabcolsep\vrule width 1.5pt\hskip\tabcolsep}}
\makeatother

\newenvironment{wideitemize}{\itemize\addtolength{\itemsep}{7pt}}{\enditemize}

\newenvironment{widelist}{
\begin{list}{}{
	\setlength{\itemsep}{7pt}
	\setlength{\parsep}{0in}
	\setlength{\parskip}{0in}
	\setlength{\topsep}{0.in}
	\setlength{\partopsep}{0in} 
	\setlength{\leftmargin}{0in}
	\setlength{\rightmargin}{0in}
	}}
{\end{list}
}

\def\MLine#1{\par\hspace*{-\leftmargin}\parbox{\textwidth}{\[#1\]}}

\makeatletter
\newcommand{\displaybump}{\hbox to \@totalleftmargin{\hfil}}
\makeatother

%\usetheme{default}
%\usecolortheme{whale}
%\useoutertheme{infolines}
%\useinnertheme{default}


%\input{macro/my_preamble}



\author[P.~Gatto]{\bf Paolo Gatto}
%\subtitle{\vspace{0.2cm}\textcolor{Black}{\it{\footnotesize Collaborator: Jan S.~Hesthaven}}}

%\collaborator{}
%\title[Low-rank compression preconditioning]{{\bf On the use of low-rank approximations for the construction of efficient preconditioners}}
\title[ddPCM]{{\bf Latest Developments of the Polarizable Continuum Model within a Domain-Decomposition Paradigm}}
\subtitle{\vspace{0.2cm}\textcolor{Black}{\it{\footnotesize \underline{Collaborators:} F.~Lipparini, B.~Stamm}}}

\date[]{{\bf MOANSI Workshop, October 12-13,  2017}}
\institute{\bf Center for Computational Engineering Science \\ RWTH Aachen}
%\institute[ICES]{\bf Institute for Computational Engineering and Science}
%\institute{\footnotesize \it Collaborators: Kyungjoo Kim, Leszek Demkowicz}
\AtBeginSection[]
{
   \begin{frame}
       \frametitle{Outline}
       \tableofcontents[currentsection]
   \end{frame}
}





\begin{document}
%\ifpdf
%\DeclareGraphicsExtensions{.pdf, .jpg, .tiff}
%\else
%\DeclareGraphicsExtensions{.eps, .jpg}
%\fi



%- TITLEPAGE ---------------------------------------------------------------
{
\usebackgroundtemplate{
\tikz\node[opacity=0.35,inner sep=0pt] {\includegraphics[height=\paperheight,width=\paperwidth]{figures/aachen_background_4}};}

\begin{frame}


\begin{center}
\includegraphics[scale=0.1]{figures/rwth_logo_1}%$\quad\:\:\:$
%%\includegraphics[scale=0.08]{figures/brown_logo2}%$\quad\:\:\:$
\end{center}

\titlepage  
%\begin{center}
%\includegraphics[scale=0.16]{figures/ICES-wordmark-teal}
%\end{center}
\end{frame}
}

%%%%%%%%%%%%%%%%%%%%%%%%%%%%%%%%%%%%%%%%%%%%%%%%%
\begin{frame}{Outline}

\tableofcontents

\end{frame}

\section{\bf Introductive Remarks}

%%%%%%%%%%%%%%%%%%%%%%%%%%%%%%%%%%%%%%%%%%%%%%%%%%%%%%%%%%%%
\begin{frame}{Why Computational Chemistry?}

\onslide*<1>{

%{\footnotesize
%\begin{center}
%\begin{table}
%	\begin{tabular}{ lrrr  }
%\toprule[0.1em] 
%{\bf Search query}	& {\bf Google Scholar hits}	& {\bf MathSciNet hits} & {\bf Ratio} \\
%\midrule[0.08em]
%\rowcolor{RedOrange}
%{\sl Navier Stokes}			& 537,000		& 8,220	& 65 \\
%\rowcolor{RedOrange}
%{\sl Maxwell equations} 		& 194,000		& 1,892	& 102 \\
%\rowcolor{YellowGreen}
%{\sl Molecular dynamics} 		& 1,550,000	& 325	& 4,769 \\
%\rowcolor{YellowGreen}
%{\sl Density functional theory} 	& 1,200,000 	& 124 	& 9,677 \\
%\rowcolor{YellowGreen}
%{\sl Particle Mesh Ewald}		& 24,200 		& 1		& 24,200 \\
%\rowcolor{YellowGreen}
%{\sl Continuum solvation model} 	& 6,320		& 0 		&$\alert{\infty}$\\
%\rowcolor{YellowGreen}
%{\sl Polarizable force field}		& 3,670		& 0		&$\alert{\infty}$ \\
%\bottomrule[0.1em]
%	\end{tabular}
%	\caption{\scriptsize October 2017}
%	\end{table}
%\end{center}
%}


{\footnotesize
\begin{center}
\begin{table}
	\begin{tabular}{ lrrr  }
\toprule[0.1em] 
{\bf Search query}	& {\bf Google Scholar hits}	& {\bf MathSciNet hits} & {\bf Ratio} \\
\midrule[0.08em]
\rowcolor{RedOrange}
{\sl Navier Stokes}			& 699,000		& 9,467	& 79 \\
\rowcolor{RedOrange}
{\sl Maxwell equations} 		& 1,120,000		& 2,648	& 423 \\
\rowcolor{YellowGreen}
{\sl Molecular dynamics} 		& 3,440,000	& 496	& 6,936 \\
\rowcolor{YellowGreen}
{\sl Density functional theory} 	& 3,050,000 	& 164 	& 18,598 \\
\rowcolor{YellowGreen}
{\sl Particle Mesh Ewald}		& 47,200 		& 2		& 23,600 \\
\rowcolor{YellowGreen}
{\sl Continuum solvation model} 	& 102,000		& 1 		&102,000\\
\rowcolor{YellowGreen}
{\sl Polarizable force field}		& 106,000		& 0		&{\normalsize${\color{Red}{\infty}}$} \\
\bottomrule[0.1em]
	\end{tabular}
	\caption{\scriptsize October 2017}
	\end{table}
\end{center}
}



}

\onslide*<2>{

\begin{figure}
\begin{center}
\includegraphics[scale=1]{figures/pie_chart.pdf}
\caption{\scriptsize Distribution of LEF's by scientific area}
\end{center}
\end{figure}

%\begin{mybox}



\begin{wideitemize}
\item {\bf Huge} usage of CPU-hours, {\bf very little} interaction with applied mathematicians
\item {\bf Great} potential for tremendous {\bf improvement}!
\end{wideitemize}


%\end{mybox}
}

\end{frame}

%%%%%%%%%%%%%%%%%%%%%%%%%%%%%%%%%%%%%%%%%%%%%%%%%%%%%%%%%%%%
\begin{frame}{Solvation Models}

\begin{figure}
\begin{center}
\includegraphics[trim={0cm 0cm 0cm 0cm}, width=0.78\textwidth]{figures/page11image9000.png}
%\caption{Example of Solvatochromism}
\end{center}
\end{figure}


\begin{wideitemize}
\item {\bf Several} relevant chemical phenomena take place in the {\bf liquid phase}
\item Effects of the {\bf environment} play a fundamental {\bf role}
\item {\bf \color{gray}Explicit} or {\bf implicit Solvation Models} to incorporate {\bf solvent effects} into numerical simulations.
\end{wideitemize}

\end{frame}


%%%%%%%%%%%%%%%%%%%%%%%%%%%%%%%%%%%%%%%%%%%%%%%%%%%%%%%%%%%%
\begin{frame}{Continuum Solvation Models}

%\vspace{-0.5cm}
%
%\begin{figure}
%\hfill
%\includegraphics[trim={2cm 1cm 2cm 3cm}, width=0.25\textwidth]{figures/Cavity1.jpg}
%\end{figure}
%
%\vfill

\begin{center}
\begin{tikzpicture}[scale=0.20]
\draw [left color = black!10 , middle color = black!5, right color=black, thin, white] (-16,-6.8) rectangle (34.3,6.3);

\node[circle, shading=ball , minimum width=1.7cm ,  ball color =white ] (ball) at (23.8,0.2) {};
\node[circle,shading=ball, minimum width=1.5cm , ball color =white] (ball) at (-1,0) {};
\node[circle,shading=ball, minimum width=1cm , ball color =white] (ball) at (3,1) {};
\node[circle,shading=ball, minimum width=1.8cm , ball color =white] (ball) at (8,0.5) {};
\node[circle,shading=ball, minimum width=1.5cm , ball color =white] (ball) at (12.5,0) {};
\node[circle, shading=ball , minimum width=2cm ,  ball color =white ] (ball) at (18.5,-0.5) {};
\node[circle,shading=ball, minimum width=1.5cm , ball color =white] (ball) at (-6,-0.5) {};
\node[circle,shading=ball, minimum width=1.2cm , ball color =white] (ball) at (-10,-1.5) {};
\node[circle, shading=ball , minimum width=0.9cm ,  ball color =white ] (ball) at (23.8,-3) {};
\end{tikzpicture}
\end{center}

\begin{wideitemize}
\item {\bf Cavity} $\Omega$, bounded by a {\bf solvent excluded surface} accommodates {\bf solute}
\item Typically, $\Omega$ is a {\bf Van der Waals} cavity
\[
\Omega = \textstyle \bigcup_{\,j=1}^{\,M} \: \Omega_j \qquad ; \qquad \Omega_j = B(x_j, r_j) \qquad , \qquad r_j = \textsl{\footnotesize Van der Waals radius} \displaybump
\]
%, i.e., union of {\bf spheres} $\Omega_j = B(x_j, r_j)$, one per atom
\item Infinite {\bf continuum} replaces individual {\bf solvent} molecules
\item Solute/solvent interaction modeled as {\bf electrostatic interaction} %between the charge density of the solute and the continuum medium
\item {\bf Short-range} interactions (dispersion, repulsion, cavitation, etc.)~treated with {\bf empirical} expressions.
\end{wideitemize}

\end{frame}


\section{\bf Conductor-like Screening Model with Domain-Decomposition}
%%%%%%%%%%%%%%%%%%%%%%%%%%%%%%%%%%%%%%%%%%%%%%%%%%%%%%%%%%%%
\begin{frame}{The Conductor-like Screening Model (COSMO)}

%\vspace{-1.2cm}
%
%\begin{figure}
%\hfill
%\includegraphics[trim={2cm 1cm 2cm 3cm}, width=0.25\textwidth]{figures/Cavity1.jpg}
%\end{figure}

%\vfill

\begin{wideitemize}
%\item COSMO approximates the solute/solvent {\bf interactions} as purely {\bf electrostatic forces}
\item {\bf Solvent} treated as {\bf conductor-like} medium with permittivity $\varepsilon_s \gg 1$
\item {\bf Electrostatic energy} of the solute/solvent system
\begin{align*}
E_s =\tfrac{1}{2} \;  f(\varepsilon_s)  \int_{\Omega}\varrho(x) W(x)\, dx \displaybump
\end{align*}
where $\varrho$ is the {\bf charge} density of the solute, $f(\varepsilon_s)$ is an empirical {\bf scaling}
\medskip
\item Poisson problem for {\bf reaction potential} $W$
\begin{alignat*}{3}
-\Delta \, W &= 0 &&\text{in } \Omega \displaybump \\
W &=-\Phi \qquad &&\text{on }\Gamma
\end{alignat*}
where $\Phi$ is the {\bf electric potential} generated by the solute in \emph{vacuum}.
\item Solute/solvent {\bf interaction} consists purely of {\bf electrostatic forces}.
\end{wideitemize}

\end{frame}


%%%%%%%%%%%%%%%%%%%%%%%%%%%%%%%%%%%%%%%%%%%%%%%%%%%%%%%%%%%%%
%\begin{frame}{The COSMO Implicit Solvation Model}
%
%
%
%\begin{wideitemize}
%
%\item Let $\varrho$ be the {\bf charge} density of the solute; let $W$ be the {\bf reaction potential} of the solvent
%\item The total {\bf electrostatic energy} of the solute/solvent system is given by
%\begin{align*}
%E_s =\tfrac{1}{2} \;  f(\varepsilon_s)  \int_{\Omega}\varrho(x) W(x)\, dx
%\end{align*}
%where $f(\varepsilon_s)$ is an empirical scaling factor
%\medskip
%\item The reaction potential $W$ satisfies the {\bf boundary value problem}
%\begin{alignat*}{3}
%-\Delta \, W &= 0 &&\text{in } \Omega \\
%W &=-\Phi \qquad &&\text{on }\Gamma
%\end{alignat*}
%where $\Phi$ is the {\bf electric potential} generated by the solute in vacuum
%\end{wideitemize}
%
%\end{frame}


%%%%%%%%%%%%%%%%%%%%%%%%%%%%%%%%%%%%%%%%%%%%%%%%%%%%%%%%%%%%%
%\begin{frame}{Solving the Laplace Equation}
%
%\begin{itemize}
%\item \textit{Naive Approach:} Use Finite Element methods.
%
%\begin{itemize}
%\item[--] Requires creating a 3-D mesh on the entire cavity $\Omega$.
%
%\item[--] FEM is usually applied to non-homogenous Elliptic PDEs. Would need to apply a lifting to Equation \eqref{eq:2}.
%\end{itemize}
%\medskip
%\item \textit{Convential Approach:} Recast Equation \eqref{eq:2} as an integral equation on the boundary $\Gamma$ and solve using a Boundary Element method.
%
%\begin{itemize}
%\item[+] Only need to mesh the boundary of the cavity.
%\item[--] Results in a \textit{dense}, ill-conditioned matrix. Even iterative methods become borderline unfeasible for very large systems.
%\end{itemize}
%\end{itemize}
%\end{frame}


%%%%%%%%%%%%%%%%%%%%%%%%%%%%%%%%%%%%%%%%%%%%%%%%%%%%%%%%%%%%
\begin{frame}{Domain-Decomposition COSMO (ddCOSMO)}

\begin{wideitemize}

\item Recast Poisson problem as {\bf integral} equation for {\bf surface charge} $\sigma$
\[
W(s) = \int_\Gamma \frac{\sigma(t)}{|s - t|} \, dt =:  (\mathcal{S} \, \sigma) (s)  \qquad \Rightarrow \qquad \mathcal{S} \, \sigma = -\Phi \qquad \text{on }\Gamma \displaybump
\]
\item Apply {\bf domain-decomposition} strategy to integral equation
\[
\mathcal{S}_j \, \sigma_j  - \sum_{\alert{k \:\: : \:\: \Omega_k \text{ neighbor of }\Omega_j }} \: \widetilde{\mathcal{S}}_{jk} \, \sigma_k = -\Phi_j \qquad \text{on }\Gamma_j \qquad ; \qquad  j = 1, \ldots , M \displaybump
\]
\item Interpret each local problem in a {\bf variational} setting
\[
\int_{\Gamma_j }\mathcal{S}_j \, \sigma_j \, \tau  - \sum_{k \, \in \,  N_j } \int_{\Gamma_j } \widetilde{\mathcal{S}}_{jk} \, \sigma_k \, \tau = - \int_{\Gamma_j }\Phi_j \, \tau \qquad \forall \, \tau \displaybump
\]
\item Expand $\sigma_j$ through {\bf spherical harmonics} on unit sphere $\mathbb{S}$
\[
\sigma_j(x_j + r_j y) = \tfrac{1}{r_j}  \sum_{\ell,m}\, [X_j]_\ell^m \, Y_\ell^{\,m}(y) \qquad , \qquad y \in \mathbb{S} \displaybump
\]
\item Select {\bf spherical harmonics} as {\bf test} functions $\tau$.
%\item Replace boundary value problem by {\bf coupled local} problems
%\begin{alignat*}{3}
%- \Delta \, W_i &= 0 &&\text{in } \Omega_i \\
%W_i &=g_i \qquad &&\text{on }\Gamma_i
%\end{alignat*}
%\item Employ {\bf iterative procedure} from initial guess $W^0$
%\begin{alignat*}{3}
%- \Delta \, {W_i}^{n+1} &= 0 &&\text{in } \Omega_i \\
%{W_i}^{n+1} &= - \Phi \qquad &&\text{on }\Gamma_i^\text{ext} \\
%{W_i}^{n+1} &= {W_j}^{n} \qquad &&\text{on }\Gamma_i^\text{int}
%\end{alignat*}

\end{wideitemize}

%\vspace{0.5cm}
%
%\begin{center}
%\begin{tikzpicture}[scale=0.25]
%\draw[] (0,0) circle [radius = 5];
%\draw[] (7,0.5) circle [radius = 5];
%\node at (0,0) {$\Omega_1$};
%\node at (7,0.5) {$\Omega_2$};
%\end{tikzpicture}
%\end{center}


\end{frame}

%%%%%%%%%%%%%%%%%%%%%%%%%%%%%%%%%%%%%%%%%%%%%%%%%%%%%%%%%%%%%
\begin{frame}{ddCOSMO Discretization}

\begin{center}
\begin{tikzpicture}[scale=0.21]
\node[circle,shading=ball, minimum width=1.5cm , ball color =white] (ball) at (-1,0) {};
\node[circle,shading=ball, minimum width=1cm , ball color =white] (ball) at (3,1) {};
\node[circle,shading=ball, minimum width=1.8cm , ball color =white] (ball) at (8,0.5) {};
\node[circle,shading=ball, minimum width=1.5cm , ball color =white] (ball) at (12.5,0) {};
\node[circle,shading=ball, minimum width=2cm , ball color =white] (ball) at (18.5,-0.5) {};
\end{tikzpicture}
\end{center}

%\begin{center}
%\begin{tikzpicture}[scale=0.23]
%%\draw [top color = olive!10 , left color = olive!10,right color = olive!10, middle color=olive , thin, white] (-13,0) rectangle (13,4.5);
%%\draw [top color = olive!10 , middle color = olive!5, bottom color=olive , thin, white] (-13,0) rectangle (13,5);
%%\draw [[top color = olive!10 , bottom color=olive , thin, white] (0,0) circle [radius = 2.5];
%\draw[fill = olive!10 , thick ] (4,1) circle [radius = 3];
%\draw[fill = olive!10 , thick] (8,0.5) circle [radius = 2];
%\draw[fill = olive!10 , thick] (12.5,0) circle [radius = 3.5];
%\draw[fill = olive!10 , thick] (17.5,-0.5) circle [radius = 2.8];
%%\draw [thick, ->] (-14,0) -- (14,0);
%%\draw [->] (0,-1) -- (0,4.5);
%%\draw [thick, ->] (-3,0) -- (-3,-2);
%\node at (0,0) {$1$};
%\node at (4,1) {$2$};
%\node at (8,0.5) {$3$};
%\node at (12.5,0) {$4$};
%\node at (17.5,-0.5) {$5$};
%%\node at (1.2,4) {$y$};
%%\node at (-4,-1) {$\boldsymbol{\nu}$};
%%\node at (-5,2.5) {$\mathbb{R}_+^{d+1}$};
%\end{tikzpicture}
%\end{center}


\begin{wideitemize}
\item Discrete operator $L$ known in {\bf closed form}, after {\bf numerical} integration
\item Operator $L$ is {\bf block-sparse}, only neighbor-to-neighbor interactions
\[ L =
{\footnotesize
\begin{pmatrix}
\times	& \times	&		& 		& \\
\times	& \times	& \times	&		& \\
		& \times	& \times	& \times	& \\
		& 		& \times	& \times 	& \times\\
		& 		&		& \times	& \times 
\end{pmatrix}
}\displaybump
\]
\item Unfortunately, operator $L$ is {\bf non symmetric}
\item Operator $L$ is {\bf well-conditioned} and {\bf diagonally dominant}
\item Employ {\bf iterative} method to solve COSMO system $L \, X = F$.
\end{wideitemize}

\end{frame}

%%%%%%%%%%%%%%%%%%%%%%%%%%%%%%%%%%%%%%%%%%%%%%%%%%%%%%%%%%%%%
\begin{frame}{How Well Are We Doing?}

\onslide*<1>{

\begin{wideitemize}
\item Compare to {\bf state-of-the-art}: Continuous Surface Charge ({\bf CSC}) method by York and Karplus
\item CSC produces {\bf smooth energy} profiles
\item CSC is {\bf not} systematically {\bf improvable}, and {\bf poorly} conditioned.
\end{wideitemize}

\vfill

\pause

{\footnotesize
\begin{center}
\begin{table}
	\begin{tabular}{ lrrrr  }
\toprule[0.1em] 
{\bf System}	& {\bf Atoms}	& {\bf CSC} {\sl(sec)} & {\bf CSC \& FMM} {\sl(sec)} & {\bf ddCOSMO} {\sl(sec)} \\
\midrule[0.08em]
{\sl Vancomycin}	& 377	& 20  	& 43 		&  \phantom{$<\,$1} \\
{\sl Hiv-1-GP41} 	& 530	& 86		& 57 		& \\
{\sl l-Plectasin} 		& 567	& 96		& 77		&  \\
{\sl Glutaredoxin} 	& 1,277 	& 534 	& 182 	& \\
{\sl UBCH5B}		& 2,360 	& 5,640	& 378 	& \\
{\sl Carboxylase} 	& 6,605	& 46,620 	& 1,482	& \\
\bottomrule[0.1em]
	\end{tabular}
		\caption{ \scriptsize {\bf Timings} for solution of CSM and ddCOSMO equations on 2*Xeon E2560 2GHz}
		\end{table}
\end{center}
}

}

\onslide*<2>{

\begin{wideitemize}
\item Compare to {\bf state-of-the-art}: Continuous Surface Charge ({\bf CSC}) method by York and Karplus
\item CSC produces {\bf smooth energy} profiles
\item CSC is {\bf not} systematically {\bf improvable}, and {\bf poorly} conditioned.
\end{wideitemize}

\vfill



{\footnotesize
\begin{center}
\begin{table}
	\begin{tabular}{ lrrrr  }
\toprule[0.1em] 
{\bf System}	& {\bf Atoms}	& {\bf CSC} {\sl(sec)} & {\bf CSC \& FMM} {\sl(sec)} & {\bf ddCOSMO} {\sl(sec)} \\
\midrule[0.08em]
{\sl Vancomycin}	& 377	& 20  	& 43 		&  \phantom{$<\,$1} \\
{\sl Hiv-1-GP41} 	& 530	& 86		& 57 		& \\
{\sl l-Plectasin} 		& 567	& 96		& 77		&  \\
{\sl Glutaredoxin} 	& 1,277 	& 534 	& 182 	& \\
{\sl UBCH5B}		& 2,360 	& 5,640	& 378 	& \\
{\sl Carboxylase} 	& 6,605	& 46,620 	& 1,482	& \\
\bottomrule[0.1em]
	\end{tabular}
		\caption{\scriptsize {\bf Timings} for solution of CSM and ddCOSMO equations on 2*Xeon E2560 2GHz}
		\end{table}
\end{center}
}

}


\onslide*<3>{

\begin{wideitemize}
\item Compare to {\bf state-of-the-art}: Continuous Surface Charge ({\bf CSC}) method by York and Karplus
\item CSC produces {\bf smooth energy} profiles
\item CSC is {\bf not} systematically {\bf improvable}, and {\bf poorly} conditioned.
\end{wideitemize}

\vfill



{\footnotesize
\begin{center}
\begin{table}
	\begin{tabular}{ lrrrr  }
\toprule[0.1em] 
{\bf System}	& {\bf Atoms}	& {\bf CSC} {\sl(sec)} & {\bf CSC \& FMM} {\sl(sec)} & {\bf ddCOSMO} {\sl(sec)} \\
\midrule[0.08em]
{\sl Vancomycin}	& 377	& 20  	& 43 		&  \cellcolor{YellowGreen} $<\,$1 \\
{\sl Hiv-1-GP41} 	& 530	& 86		& 57 		& \cellcolor{YellowGreen}$<\,$1 \\
{\sl l-Plectasin} 		& 567	& 96		& 77		& \cellcolor{YellowGreen}$<\,$1 \\
{\sl Glutaredoxin} 	& 1,277 	& 534 	& 182 	& \cellcolor{YellowGreen}$<\,$1 \\
{\sl UBCH5B}		& 2,360 	& 5,640	& 378 	& \cellcolor{YellowGreen}1 \\
{\sl Carboxylase} 	& 6,605	& 46,620 	& 1,482	& \cellcolor{YellowGreen}3 \\
\bottomrule[0.1em]
	\end{tabular}
		\caption{\scriptsize {\bf Timings} for solution of CSM and ddCOSMO equations on 2*Xeon E2560 2GHz}
		\end{table}
\end{center}
}

}


\end{frame}


\section{\bf Polarizable Continuum Model with Domain-Decomposition}
%%%%%%%%%%%%%%%%%%%%%%%%%%%%%%%%%%%%%%%%%%%%%%%%%%%%%%%%%%%%%
\begin{frame}{Polarizable Continuum Model (PCM)}

\begin{wideitemize}

\item {\bf Solvent} treated as a {\bf dielectric} medium with finite permittivity $\varepsilon_s$ 

\item {\bf Modified} Poisson problem for {reaction potential} $W$
\begin{alignat*}{3}
- \dive ( \varepsilon \, \nabla W ) &= 0 &&\text{in } \mathbb{R}^3 \setminus \Gamma \displaybump \\
[\![W ]\!]&=0 \qquad &&\text{on }\Gamma \\
[\![ \varepsilon \, \partial_{\bnu} W ]\!] &=(\varepsilon_s - 1 ) \, \partial_{\bnu} \Phi \qquad &&\text{on }\Gamma
\end{alignat*}
where $[\![ \cdot ]\!]$ is {\bf jump} operator, and $\varepsilon(x) = 1$ if $x \in \Omega$, and $\varepsilon(x) = \varepsilon_s$ otherwise

\item Recast as {\bf integral} equation
\[
\mathcal{R}_\varepsilon \, \mathcal{S} \, \sigma = -\mathcal{R}_\infty \, \Phi \qquad \text{on }\Gamma \displaybump
\]
where $\mathcal{R}_\varepsilon = 2\pi \,g(\varepsilon_s) \, Id - \mathcal{D}$, and $\mathcal{D}$ is the {\bf double layer} operator
\item Solve as a {\bf two-step} problem involving {COSMO}
\[
\mathcal{R}_\varepsilon \, \Phi_\varepsilon =  \mathcal{R}_\infty \, \Phi \qquad \text{on } \Gamma \qquad ; \qquad
\mathcal{S} \, \sigma =  -\Phi_\varepsilon \qquad \text{on } \Gamma \displaybump
\]

\end{wideitemize}

\end{frame}

%%%%%%%%%%%%%%%%%%%%%%%%%%%%%%%%%%%%%%%%%%%%%%%%%%%%%%%%%%%%%
\begin{frame}{Domain-Decomposion PCM (ddPCM)}

\begin{wideitemize}
\item Let $\Phi_j\: , \: \Phi_{\varepsilon,j} : \Gamma_j \to \mathbb{R}$ be the {\bf trivial extensions} of $\Phi$ and $\Phi_{\varepsilon}$ to $\Gamma_j$
\item Decompose {\bf double layer} operator as sum of {\bf local} contributions
\[
(\mathcal{D} \, \Phi_{\color{Gray}{\varepsilon}})(s) = (\mathcal{D}_j \, \Phi_{{\color{Gray}{\varepsilon,}}j})(s) + \sum_{k \, \ne \, j} \, (\widetilde{\mathcal{D}}_k \, \Phi_{{\color{Gray}{\varepsilon,}}k})(s) \qquad , \qquad s \in \Gamma_j^\text{ext} := \Gamma \cap \Gamma_j \displaybump
\]
\item {\bf Localize} integral equation through {\bf characteristic} function $U_j$ of $\Gamma_j^\text{ext}$
\[
2 \pi \, g(\varepsilon_s) \, \Phi_{\varepsilon,j} - U_j \, \mathcal{D}\, \Phi_\varepsilon = 2 \pi \, \Phi_j - U_j \mathcal{D} \,\Phi \qquad \text{on }\Gamma_j \displaybump
\]
\item Apply {\bf decomposition} to obtain final form
\[
2\pi \, g(\varepsilon_s) \, \Phi_{\varepsilon,j} - U_j \bigg( {\mathcal{D}}_j \, \Phi_{\varepsilon,j} + \sum_{\alert{k \, \ne \, j}} \, \widetilde{\mathcal{D}}_{k} \, \Phi_{\varepsilon,k}  \bigg) = \textsl{\footnotesize analogous to LHS} %\\ 2 \pi \, {\Phi_j} - U_j \bigg( {\mathcal{D}}_j \,\Phi_{j} + \sum_{k \ne j} \, \tilde{\mathcal{D}}_{k} \, \Phi_{k}  \bigg) 
\qquad \text{on }\Gamma_j  \displaybump
\]
\item Interpret local problem in a {\bf variational} setting as for COSMO. %, expand $\Phi_{\varepsilon,j}$ thorough {\bf spherical harmonics}, select spherical harmonics as {\bf test} functions.

\end{wideitemize}

\end{frame}

%%%%%%%%%%%%%%%%%%%%%%%%%%%%%%%%%%%%%%%%%%%%%%%%%%%%%%%%%%%%%
\begin{frame}{ddPCM Discretization}

\begin{center}
\begin{tikzpicture}[scale=0.21]
\node[circle,shading=ball, minimum width=1.5cm , ball color =white] (ball) at (-1,0) {};
\node[circle,shading=ball, minimum width=1cm , ball color =white] (ball) at (3,1) {};
\node[circle,shading=ball, minimum width=1.8cm , ball color =white] (ball) at (8,0.5) {};
\node[circle,shading=ball, minimum width=1.5cm , ball color =white] (ball) at (12.5,0) {};
\node[circle,shading=ball, minimum width=2cm , ball color =white] (ball) at (18.5,-0.5) {};
\end{tikzpicture}
\end{center}

%\begin{center}
%\begin{tikzpicture}[scale=0.23]
%%\draw [top color = olive!10 , left color = olive!10,right color = olive!10, middle color=olive , thin, white] (-13,0) rectangle (13,4.5);
%%\draw [top color = olive!10 , middle color = olive!5, bottom color=olive , thin, white] (-13,0) rectangle (13,5);
%%\draw [[top color = olive!10 , bottom color=olive , thin, white] (0,0) circle [radius = 2.5];
%\draw[fill = olive!10 , thick ] (4,1) circle [radius = 3];
%\draw[fill = olive!10 , thick] (8,0.5) circle [radius = 2];
%\draw[fill = olive!10 , thick] (12.5,0) circle [radius = 3.5];
%\draw[fill = olive!10 , thick] (17.5,-0.5) circle [radius = 2.8];
%%\draw [thick, ->] (-14,0) -- (14,0);
%%\draw [->] (0,-1) -- (0,4.5);
%%\draw [thick, ->] (-3,0) -- (-3,-2);
%\node at (0,0) {$1$};
%\node at (4,1) {$2$};
%\node at (8,0.5) {$3$};
%\node at (12.5,0) {$4$};
%\node at (17.5,-0.5) {$5$};
%%\node at (1.2,4) {$y$};
%%\node at (-4,-1) {$\boldsymbol{\nu}$};
%%\node at (-5,2.5) {$\mathbb{R}_+^{d+1}$};
%\end{tikzpicture}
%\end{center}


\begin{wideitemize}
\item Discrete operator $A_\varepsilon$ known in {\bf closed form}, after {\bf numerical} integration
\item Operator $A_\varepsilon$ is {\bf dense}, includes {\bf long-range} interactions
\[ A_\varepsilon =
{\footnotesize
\begin{pmatrix}
\times				& \times					& \alert{\boldsymbol{\times}}	& \alert{\boldsymbol{\times}} 	& \alert{\boldsymbol{\times}} \\
\times				& \times					& \times					& \alert{\boldsymbol{\times}} 	& \alert{\boldsymbol{\times}} \\
\alert{\boldsymbol{\times}}	& \times					& \times					& \times 					& \alert{\boldsymbol{\times}} \\
\alert{\boldsymbol{\times}}	& \alert{\boldsymbol{\times}}	& \times					& \times 					& \times \\
\alert{\boldsymbol{\times}}	& \alert{\boldsymbol{\times}}	& \alert{\boldsymbol{\times}}	& \times 					& \times 
\end{pmatrix}
}  \displaybump
\]
\item Unfortunately, operator $A_\varepsilon$ is {\bf non symmetric}
\item Employ {\bf Jacobi iteration} with {\bf DIIS} to solve ddPCM system $A_\varepsilon \, G = A_\infty \, F$.
%\item Operator $L$ is {\bf well-conditioned} and {\bf diagonally dominant}
%\item Employ {\bf iterative} method to solve COSMO system $L \, X = F$.
\end{wideitemize}


\end{frame}


%%%%%%%%%%%%%%%%%%%%%%%%%%%%%%%%%%%%%%%%%%%%%%%%%%%%%%%%%%%%%
\begin{frame}{ddPCM Forces}

\begin{wideitemize}
\item Individual {\bf domain contributions} to solvation energy
\[
E_s = \tfrac{1}{2} \, f(\varepsilon_s) \int_\Omega \varrho \, W =  \tfrac{1}{2} \, f(\varepsilon_s) \: {\textstyle \sum_{\, j}} \int_{\Omega_j} \varrho \, (\chi_j \, W) %= \cdots =  \tfrac{1}{2} \, f(\varepsilon_s)  \, \langle \Psi , X \rangle
%\: {\textstyle \sum_{\, j}} \,  {\textstyle \sum_{\, \ell,m}} \, [\Psi_j]_\ell^m \,  [X_j]_\ell^m
 \displaybump
\]
through a {\bf partition of unity} $\{\chi_j\}$
\item Approximation of {\bf local} potentials
\[
W_j (x) = (\widetilde{\mathcal{S}_j} \, \sigma_j)(x) = \sum_{\ell,m} \, [X_j]_\ell^m \, (\widetilde{\mathcal{S}}_{\, \mathbb{S}} \, Y_\ell^{\,m})(u)
% \underset{\ell,m}{\textstyle \sum} \: \tfrac{4 \pi }{2\ell + 1}\, |u|^\ell \, Y_\ell^m(u / |u|) \, [X_j]_\ell^m 
\qquad , \qquad x = x_j + r_j u  \displaybump
\]
\item Express energy as double {\bf scalar product}
\[
 E_s =\tfrac{1}{2} \,  f(\varepsilon_s) \, \langle   \Psi \, , X \rangle \qquad , \qquad \langle \cdot \, , \cdot \rangle :=  \sum_j \, \sum_{\ell,m} \, \cdots  \displaybump
 \]
where $\Psi$ is a vector {\bf independent} of nuclear positions
\item {\bf Force} acting on $i$-th atom
\[
\mathcal{F}_i = - \nabla E_s = - \tfrac{1}{2} \, f(\varepsilon_s)  \, \langle \Psi \, , \nabla X \rangle \qquad \textsl{\footnotesize gradient with respect to } x_i  \displaybump
\]
%$\langle \cdot \, , \cdot \rangle $ is a double scalar product, and $\Psi$

\end{wideitemize}

\end{frame}


%%%%%%%%%%%%%%%%%%%%%%%%%%%%%%%%%%%%%%%%%%%%%%%%%%%%%%%%%%%%%
\begin{frame}{How to Compute Forces?}

\onslide*<1>{

\vspace{0.3cm}

\begin{wideitemize}

\item {\bf Adjoint} problem $(A_\varepsilon \, L)' s = \Psi$ {\bf trasfers} operator to second argument
\[
\langle \Psi \, , \nabla X \rangle = \langle (A_\varepsilon \, L)' s \, , \nabla X \rangle = \langle  s \, , A_\varepsilon \, L \,\nabla X \rangle \displaybump
\]

\item {\bf Integration-by-parts} like approach through Leibniz rule
\begin{alignat*}{4}
A_\varepsilon \, \alert{G}& = A_\infty \, F  \qquad && \Rightarrow \qquad &  A_\varepsilon \, \alert{\nabla G} & = \nabla A ( F -G) + A_\infty \, \nabla F \displaybump \\
L \, \alert{X} & = \alert{G} &&\Rightarrow  &L \, \alert{\nabla X} & = \alert{\nabla G} - \nabla L \,  X
\end{alignat*}
\item Put it {\bf all together}, recalling intermediate solves $L' \, y= \Psi$ and $A_\varepsilon' \, s = y$
\begin{multline*}
\langle  s \, , A_\varepsilon \, L \,\alert{\nabla X} \rangle = \langle  s \, , A_\varepsilon \, \alert{\nabla G} \rangle -\langle  s \, ,A_\varepsilon \,  \nabla L \,  X  \rangle = \\
\phantom{A_\varepsilon'}= \langle  s \, ,\nabla A ( F - G) \rangle + \langle  s \, , A_\infty \, \nabla F \rangle -\langle  s \, , A_\varepsilon \, \nabla L \,  X  \rangle \displaybump
\end{multline*}
\item Since $A_\infty' \, s = (A_\infty - A_\varepsilon)' s + y$, write as {\bf perturbation} of COSMO
\[
\langle \Psi \, , \nabla X \rangle = \langle  s \, ,\nabla A ( F - G) \rangle + \langle \underbracket{(A_\infty - A_\varepsilon)'}_{\textsl{diagonal}} s \, ,  \nabla F \rangle + \underbracket{\langle y \, , \nabla F \rangle - \langle  y \, ,  \nabla L \,  X  \rangle}_{\textsl{COSMO} \:\: , \:\: O(1)} \displaybump
\]

\end{wideitemize}

}


\onslide*<2>{

\vspace{0.3cm}

\begin{wideitemize}

\item {\bf Adjoint} problem $(A_\varepsilon \, L)' s = \Psi$ {\bf trasfers} operator to second argument
\[
\langle \Psi \, , \nabla X \rangle = \langle (A_\varepsilon \, L)' s \, , \nabla X \rangle = \langle  s \, , A_\varepsilon \, L \,\nabla X \rangle \displaybump
\]

\item {\bf Integration-by-parts} like approach through Leibniz rule
\begin{alignat*}{4}
A_\varepsilon \, \alert{G}& = A_\infty \, F  \qquad && \Rightarrow \qquad &  A_\varepsilon \, \alert{\nabla G} & = \nabla A ( F -G) + A_\infty \, \nabla F \displaybump \\
L \, \alert{X} & = \alert{G} &&\Rightarrow  &L \, \alert{\nabla X} & = \alert{\nabla G} - \nabla L \,  X
\end{alignat*}
\item Put it {\bf all together}, recalling intermediate solves $L' \, y= \Psi$ and $A_\varepsilon' \, s = y$
\begin{multline*}
\langle  s \, , A_\varepsilon \, L \,\alert{\nabla X} \rangle = \langle  s \, , A_\varepsilon \, \alert{\nabla G} \rangle -\langle  s \, ,A_\varepsilon \,  \nabla L \,  X  \rangle = \\
= \langle  s \, ,\nabla A ( F - G) \rangle + \langle A_\infty' \,  s \, ,  \nabla F \rangle -\langle A_\varepsilon' \,  s \, , \nabla L \,  X  \rangle \displaybump
\end{multline*}
\item Since $A_\infty' \, s = (A_\infty - A_\varepsilon)' s + y$, write as {\bf perturbation} of COSMO
\[
\langle \Psi \, , \nabla X \rangle = \langle  s \, ,\nabla A ( F - G) \rangle + \langle \underbracket{(A_\infty - A_\varepsilon)'}_{\textsl{diagonal}} s \, ,  \nabla F \rangle + \underbracket{\langle y \, , \nabla F \rangle - \langle  y \, ,  \nabla L \,  X  \rangle}_{\textsl{COSMO} \:\: , \:\: O(1)} \displaybump
\]

\end{wideitemize}

}

\onslide*<3>{

\vspace{0.3cm}

\begin{wideitemize}

\item {\bf Adjoint} problem $(A_\varepsilon \, L)' s = \Psi$ {\bf trasfers} operator to second argument
\[
\langle \Psi \, , \nabla X \rangle = \langle (A_\varepsilon \, L)' s \, , \nabla X \rangle = \langle  s \, , A_\varepsilon \, L \,\nabla X \rangle \displaybump
\]

\item {\bf Integration-by-parts} like approach through Leibniz rule
\begin{alignat*}{4}
A_\varepsilon \, \alert{G}& = A_\infty \, F  \qquad && \Rightarrow \qquad &  A_\varepsilon \, \alert{\nabla G} & = \nabla A ( F -G) + A_\infty \, \nabla F \displaybump \\
L \, \alert{X} & = \alert{G} &&\Rightarrow  &L \, \alert{\nabla X} & = \alert{\nabla G} - \nabla L \,  X
\end{alignat*}
\item Put it {\bf all together}, recalling intermediate solves $L' \, y= \Psi$ and $A_\varepsilon' \, s = y$
\begin{multline*}
\langle  s \, , A_\varepsilon \, L \,\alert{\nabla X} \rangle = \langle  s \, , A_\varepsilon \, \alert{\nabla G} \rangle -\langle  s \, ,A_\varepsilon \,  \nabla L \,  X  \rangle = \\
= \langle  s \, ,\nabla A ( F - G) \rangle + \langle A_\infty' \,  s \, ,  \nabla F \rangle -\langle \phantom{A_\varepsilon' } y \, , \nabla L \,  X  \rangle \displaybump
\end{multline*}
\item Since $A_\infty' \, s = (A_\infty - A_\varepsilon)' s + y$, write as {\bf perturbation} of COSMO
\[
\langle \Psi \, , \nabla X \rangle = \langle  s \, ,\nabla A ( F - G) \rangle + \langle \underbracket{(A_\infty - A_\varepsilon)'}_{\textsl{diagonal}} s \, ,  \nabla F \rangle + \underbracket{\langle y \, , \nabla F \rangle - \langle  y \, ,  \nabla L \,  X  \rangle}_{\textsl{COSMO} \:\: , \:\: O(1)} \displaybump
\]

\end{wideitemize}

}


\end{frame}


%%%%%%%%%%%%%%%%%%%%%%%%%%%%%%%%%%%%%%%%%%%%%%%%%%%%%%
\begin{frame}{Derivatives of ddPCM}

\onslide*<1>{

\begin{wideitemize}

\item Arrange computations of {\bf contraction product} as
\[
\langle s \, , \nabla A \, z \rangle = \underbracket{ s_i \sum_{\phantom{j}k\phantom{j}} \nabla_{\!i} \,A_{ik} \, z_k}_{O(M)} + \underbracket{\bigg( \sum_{j \, \ne \, i} s_j \, \nabla_{\!i} \,A_{ji}  \bigg) z_i}_{O(M)} + \underbracket{\sum_{k \, \ne \, i}\bigg(\: \sum_{j \, \ne \, i} \: s_j \, \nabla_{\!i} \,A_{ji} \, z_k \bigg) z_k}_{O(M^2)} \displaybump
\]


\item Exploit {\bf sparsity} of derivative to {\bf accelerate} computations
\begin{alignat*}{5}
\nabla_{\!i} \, A_{jj} & \quad \textsl{\footnotesize a priori nonzero} \qquad && \Leftarrow \qquad  j && \in N_i  &&\vee && j = i \\
\nabla_{\!i} \, A_{jk} & \quad \textsl{\footnotesize a priori nonzero} \qquad && \Leftarrow \qquad j && \in N_i  \quad  &&\vee  \quad && j = i \quad \vee \quad k = i \displaybump
\end{alignat*}
\[
\nabla_{\!1} \, A = {\footnotesize
\begin{pmatrix}
\times	& \times	& \times	& \times	& \times \\
\times	& \times	& \times	& \times	& \times \\
\times	&		&	&	& \\
\times	&		&	&	& \\
\times	&		&	&	&
\end{pmatrix} 
} \quad ; \quad 
\nabla_{\!2} \, A = {\footnotesize
\begin{pmatrix}
\times	& \times	& \times	& \times	& \times \\
\times	& \times	& \times	& \times	& \times \\
\times	& \times	& \times	& \times	& \times \\
		& \times	&		&		& \\
		& \times	&		&		&
\end{pmatrix}
} \quad \cdots \displaybump
\]

\pause

\item Computation of $M$ forces has {\bf quadratic} complexity.% with {\bf preconstant} depending on $L_\text{max}$, number of integration points, etc

\end{wideitemize}

}

\onslide*<2>{

\begin{wideitemize}

\item Arrange computations of {\bf contraction product} as
\[
\langle s \, , \nabla A \, z \rangle = \underbracket{ s_i \sum_{\phantom{j}k\phantom{j}} \nabla_{\!i} \,A_{ik} \, z_k}_{O(M)} + \underbracket{\bigg( \sum_{j \, \ne \, i} s_j \, \nabla_{\!i} \,A_{ji}  \bigg) z_i}_{O(M)} + \underbracket{\sum_{k \, \ne \, i}\bigg( \sum_{\alert{j \, \in \, N_i}} s_j \, \nabla_{\!i} \,A_{ji} \, z_k \bigg) z_k}_{\alert{\# \, \textsl{\footnotesize neighbors} \: \times \: O(M)}} \displaybump
\]


\item Exploit {\bf sparsity} of derivative to {\bf accelerate} computations
\begin{alignat*}{5}
\nabla_{\!i} \, A_{jj} & \quad \textsl{\footnotesize a priori nonzero} \qquad && \Leftarrow \qquad  j && \in N_i  &&\vee && j = i \\
\nabla_{\!i} \, A_{jk} & \quad \textsl{\footnotesize a priori nonzero} \qquad && \Leftarrow \qquad j && \in N_i  \quad  &&\vee  \quad && j = i \quad \vee \quad k = i \displaybump
\end{alignat*}
\[
\nabla_{\!1} \, A = {\footnotesize
\begin{pmatrix}
\times	& \times	& \times	& \times	& \times \\
\times	& \times	& \times	& \times	& \times \\
\times	&		&	&	& \\
\times	&		&	&	& \\
\times	&		&	&	&
\end{pmatrix} 
} \quad ; \quad 
\nabla_{\!2} \, A = {\footnotesize
\begin{pmatrix}
\times	& \times	& \times	& \times	& \times \\
\times	& \times	& \times	& \times	& \times \\
\times	& \times	& \times	& \times	& \times \\
		& \times	&		&		& \\
		& \times	&		&		&
\end{pmatrix}
} \quad \cdots \displaybump
\]


\item Computation of $M$ forces has {\bf quadratic} complexity.% with {\bf preconstant} depending on $L_\text{max}$, number of integration points, etc

\end{wideitemize}

}


\end{frame}
\section{{\bf Conclusions}}
%\section{{\bf Conclusions \& Outlook}}


%%%%%%%%%%%%%%%%%%%%%%%%%%%%%%%%%%%%%%%%%%%%%%%%%%%%%%%
\begin{frame}{Conclusions}

%\begin{mybox}
\begin{wideitemize}
\item {\bf Novel} method for discretization of {\bf Polarizable Continuum Model}
\item Built on {\bf spherical geometry}, {\bf integral equations}, and {\bf spherical harmonics}
\item Method can be accelerated to {\bf linear scaling} employing {\bf FMM}
\item Coupling with {\bf Molecular Dynamics} code
\item
\end{wideitemize}
%\end{mybox}


\vfill

{\footnotesize
{P.~Gatto}, F.~Lipparini, B.~Stamm. ``Computation of Forces arising from the Polarizable Continuum Model within the Domain-Decomposition Paradigm.'' Submitted to \textit{Journal of Chemical Physics}.
}
\end{frame}

%%%%%%%%%%%%%%%%%%%%%%%%%%%%%%%%%%%%%%%%%%%%%%%%%%%%%%%%
%\begin{frame}{Outlook}
%
%
%\begin{mybox}
%\begin{wideitemize}
%
%\item Develop a strategy to automatically construct a competitive partitioning through a graph partitioner the employs grid information;
%\item Further investigate the applicability of the preconditioner: different types of discretizations, time-varying problems, etc.;
%
%\item Study the resilience of the preconditioner.
%
%\end{wideitemize}
%\end{mybox}
%
%\end{frame}


%%%%%%%%%%%%%%%%%%%%%%%%%%%%%%%%%%%%%%%%%%%%%%%%%%%%%%%%
%\begin{frame}{Computational Fluid Dynamics}
%
%\begin{columns}
%\begin{column}[c]{5cm}
%\begin{center}
%\includegraphics[width = 1\textwidth]{figures/cfd.eps}\\
%\texttt{Pres\_Poisson.mat}
%\end{center}
%\end{column}
%
%\begin{column}[c]{5cm}
%\begin{center}
%\includegraphics[width = 1\textwidth]{figures/imatrix_cfd.eps}
%\end{center}
%\end{column}
%
%\end{columns}
%
%
%\end{frame}





%%%%%%%%%%%%%%%%%%%%%%%%%%%%%%%%%%%%%%%%%%%%%%%%%%%%%%%
\section{}
{
\usebackgroundtemplate{
%\tikz\node[opacity=0.45,inner sep=0pt] {\includegraphics[height=\paperheight,width=\paperwidth]{figures/epfl_photo}};}
\tikz\node[opacity=0.35,inner sep=0pt] {\includegraphics[height=\paperheight,width=\paperwidth]{figures/aachen_background_4}};}
\begin{frame}{}

\vspace{1cm}
\begin{center}
\huge{\bf Thank you!} \\
\vspace{0.5cm}
\large{Questions?}
\end{center}
\end{frame}
}






\end{document}
