\section{Conclusions}\label{sec:conclusions}
We presented analytical gradients of the Polarizable Continuum Model electrostatic solvation energy computed using the domain decomposition strategy. The complete derivation is offered in this paper, and a correctly scaling implementation is discussed. Numerical experiments prove not only the correctness of the implementation via the comparison of analytical and numerical derivatives, but also the predicted scaling of the computational cost with respect to the number of atoms of our implementation.
The pilot implementation presented in this work is a starting point for future work. While the overall timings here presented are acceptable and compare well with the ones obtained with different discretizations, the computational effort required by the solution of the ddPCM equations and the computation of the ddPCM forces is still large, and makes routine applications on large and very large systems not pursuable. In particular, while a more efficient implementation can be achieved and a better parallelization strategy implemented, the quadratic scaling of the computation represents a barrier that cannot be circumvented. For this reason, a linear scaling implementation based on the use of the Fast Multipole Method is currently under investigation. Furthermore, we have presented results for classical solutes only, i.e., solutes described by a classical force field. An implementation of ddPCM in the framework of quantum chemical methods is particularly attractive. ddPCM shares with ddCOSMO the rigorous foundations, variational discretization and overall simplicity and limited number of parameters. Furthermore, even the quadratic scaling code has a smaller constant than BEM-based discretization (which scale quadratically in the number of total grid points, i.e., like $\mathcal{O}(M^2N^2_g)$, to be compared with $\mathcal{O}(M^2N_g)$ for ddPCM) and is expected to be competitive. As the systems that can be treated with quantum mechanical methods are well within the sizes accessible by ddPCM, we expect the method to be already fully applicable in the framework of quantum chemistry. 
