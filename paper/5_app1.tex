\section{ddPCM discretization\label{app:pcm}}
The derivation of the ddPCM discretization employs the fact that the spherical harmonics $Y_\ell^m$ are eigenfunctions of the double layer operator $\cD$ on the unit sphere $\mathbb{S}$, i.e., $\cD \, Y_\ell^m =  -2\pi/ (2\ell+1) \,  Y_\ell^m$, along with the following jump relation for the double layer potential
\begin{equation}\label{eq:jump}
	\lim_{\delta \to + 0} \big(\tilde{\cD} \, Y_\ell^m\big)(y \pm \delta \nu) =  \pm 2\pi \, Y_\ell^m(y)+ ( \cD \, Y_\ell^m )(y)
\end{equation}
where $\nu$ denotes the outward normal at $y \in \mathbb{S}$. We shall also rely on the invariance by translation and scaling of the double layer operator, namely  $(\mathcal{D}_j \, \Phi_{\varepsilon,j})(s) = (\mathcal{D} \, \hat{\Phi}_{\varepsilon,j})(y)$, where $y = (s - x_j)/r_j$ and $\hat{\Phi}_{\varepsilon,j}$ is defined on $\mathbb{S}$ through the push-forward-like transformation $\hat{\Phi}_{\varepsilon,j}(y) = \Phi_{\varepsilon,j}(s)$. An analogous result holds for the double layer potential, namely $(\tilde{\cD}_k \, \Phi_{\varepsilon,k} )(x) = (\tilde{\cD} \, \hat{\Phi}_{\varepsilon,j})(u)$ where $x \in \mathbb{R}^3 \setminus \overline{\Omega}_k$, $u = (x -x_j)/ r_j$, and $\tilde{\cD}$ is the double layer potential on $\mathbb{S}$.

As customary, in order to obtain a numerical discretization of \eqref{eq:1}, we multiply by a test function $\varphi$ and integrate over $\Gamma_j$
\begin{multline*}
\int_{\Gamma_j}\big( 2\pi \, g(\varepsilon_s) \, \Phi_{\varepsilon,j} - U_j \, {\mathcal{D}}_j \, \Phi_{\varepsilon,j} \big) \varphi +  \sum_{k \ne j} \, \int_{\Gamma_j} U_j \, \tilde{\mathcal{D}}_{k} \, \Phi_{\varepsilon,k} \, \varphi  = \\ 
= \int_{\Gamma_j}\big( 2 \pi \, {\Phi_j} - U_j \, {\mathcal{D}}_j \,\Phi_{j} \big) \varphi + \sum_{k \ne j} \,\int_{\Gamma_j } U_j \,  \tilde{\mathcal{D}}_{k} \, \Phi_{k}  \, \varphi
\end{multline*}
Here we isolated the diagonal terms for convenience.  Through the change of variable $y = (s- x_j)/r_j$, we obtain integrals over $\mathbb{S}$ which involve the hatted quantities, namely
\begin{multline}\label{eq:73}
 \int_{\mathbb{S}}\big( 2\pi \, g(\varepsilon_s) \, \hat{\Phi}_{\varepsilon,j} - \hat{U}_j \, {\mathcal{D}} \, \hat{\Phi}_{\varepsilon,j} \big) \hat{\varphi} + \sum_{k \ne j} \,\int_{\mathbb{S}}  \hat{U}_j \, \tilde{\mathcal{D}} \, \hat{\Phi}_{\varepsilon,k} \, \hat{\varphi}  = \\ 
= \int_{\mathbb{S}}\big( 2 \pi \, \hat{\Phi}_j - \hat{U}_j \, {\mathcal{D}} \,\hat{\Phi}_j \big) \hat{\varphi} + \sum_{k \ne j} \, \int_{\mathbb{S}}  \hat{U}_j \, \tilde{\mathcal{D}} \, \hat{\Phi}_{k}  \, \hat{\varphi}
\end{multline}
where we divided through by the surface Jacobian $r_j^2$.

In order to discretize the left-hand-side of the previous equation, we expand $\hat{\Phi}_{\varepsilon,j}$ as a series of spherical harmonics with coefficients $-[G_j]_{\ell'}^{m'}$, and select as a test function $\hat{\varphi}$  the spherical harmonic $Y_{\ell}^{m}$. For the diagonal term, the orthogonality of the spherical harmonics, together with the fact that they are eigenfunctions of the double layer potential, yield
\begin{multline*}
 \int_{\mathbb{S}}\big( 2\pi \, g(\varepsilon_s) \, \hat{\Phi}_{\varepsilon,j} - \hat{U}_j \, {\mathcal{D}} \, \hat{\Phi}_{\varepsilon,j} \big) Y_\ell^m = \\
= - 2 \pi  \, g(\varepsilon_s) \, \sum_{\ell',m'}  \, [G_j]_{\ell'}^{m'} \, \delta_{\ell \ell'} \delta_{mm'}  - \sum_{\ell',m'} \, [G_j]_{\ell'}^{m'} \, \frac{2\pi}{2\ell'+1} \int_{\mathbb{S}} \hat{U}_j \,  Y_{\ell'}^{m'}\, Y_\ell^m
\end{multline*}
The last step to obtain the diagonal block $A_{jj}^\varepsilon$ is to approximate the integral through a suitable quadrature formula with weights $\{w_n\}$ and nodes $\{ s_n\}$. Once the numerical quadrature is carried out and the spherical harmonics expansion is truncated at $\ell' = L_\text{max}$, we derive
\begin{equation}\label{eq:ajj}
[A_{jj}^\varepsilon]_{\ell \ell'}^{mm'} = 2\pi \, g(\varepsilon_s) %\frac{\varepsilon+1}{\varepsilon-1}
 \,  \delta_{\ell \ell'} \delta_{mm'} + \frac{2\pi}{2\ell'+1} \sum_{n=1}^{N_\text{grid}} \, w_n \, \hat{U}_j(s_n)  \, Y_{\ell'}^{m'}(s_n)\,  Y_\ell^m(s_n)
\end{equation}
which is the final expression of the diagonal block.

In order to discuss the off-diagonal blocks, let us write
\[
 \int_{\mathbb{S}} \hat{U}_j(y) \, (\tilde{\cD} \, \hat{\Phi}_{\varepsilon,k})(u(y)) \, Y_{\ell}^{m}(y) \, dy = - \sum_{\ell',m'} \, [G_k]_{\ell'}^{m'} \,  \int_{\mathbb{S}} \hat{U}_j(y) \, (\tilde{\cD} \, Y_{\ell'}^{m'})(u(y)) \, Y_{\ell}^{m}(y) \, dy 
\]
where we have explicitly indicated the variable of integration. We recall that, when $x \in \Gamma_j$, i.e., $x = s = x_j + r_j y$ for some $y \in \mathbb{S}$, then $u = u(y) = (x_j + r_j y -x_k)/r_k$. The function $\tilde{\cD} \, Y_{\ell'}^{m'}$ is harmonic on $\mathbb{R}^3 \setminus \overline{B(0,1)}$, so that is has to coincide with the unique harmonic extension of its boundary value. The jump relation \eqref{eq:jump}, along with the eigenfunction property, provide the boundary value
\[
\lim_{\delta \; \to \; + 0} \big(\tilde{\cD} \, Y_{\ell'}^{m'}\big)(y + \delta \nu) =  2\pi \,Y_{\ell'}^{m'}(y)+ \big( \cD \, Y_{\ell'}^{m'} \big)(y) = \frac{4 \pi \ell'}{2 \ell +' 1} \, Y_{\ell'}^{m'}(y)
\]
and, by elementary notions on harmonic functions, we conclude
\[
(\tilde{\cD} \, Y_{\ell'}^{m'})(u) = \frac{4 \pi \ell'}{2 \ell' + 1} \, \frac{1}{|u|^{\ell' + 1}} \,  Y_{\ell'}^{m'}(u/|u|)
\]
After truncation the series expansion and performing numerical integration we obtain
\begin{equation}\label{eq:ajk}
[A_{jk}]_{\ell \ell'}^{m m'} =-  \frac{4 \pi \ell'}{2 \ell' + 1}  \sum_{n=1}^{N_\text{grid}} \, w_n  \,  \hat{U}_j(s_n) \, \frac{1}{|u(s_n)|^{\ell' + 1}} \, Y_{\ell'}^{m'}(u(s_n) / |u(s_n)|) \, Y_{\ell}^{m}(s_n)
\end{equation}
which is the final result.


The discretization of the right-hand-side of the ddPCM equation \eqref{eq:73} is entirely analogous. As a first step, we expand $\Phi_j$ as a series of spherical harmonics, recall \eqref{eq:71}, so that the orthogonality condition yields the following expression
\[
[F_j]_\ell^m = - \int_{\mathbb{S}} \hat{\Phi}_j \, Y_\ell^m= - \int_{\mathbb{S}} \hat{U}_j \, \hat{\Phi} \, Y_\ell^m(y)
\]
which is evaluated through numerical quadrature. The derivation of the discrete operator $A_\infty$ arising from the right-hand-side is entirely analogous to the one described for the left-hand-side. In fact, it coincides with $A_\varepsilon$ with the natural extension $g(\infty) = 1$.